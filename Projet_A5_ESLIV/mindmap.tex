% Use the standalone as the document class with an option which allows specifying the border.
\documentclass[border = 10pt]{standalone}

% Allow to entering comfortably English.
\usepackage[T1]{fontenc}
\usepackage[utf8]{inputenc}
\usepackage[french, english]{babel}


% 
\usepackage{xcolor}
\definecolor{color0}{HTML}{372639}
\definecolor{color1}{HTML}{C2A4C0}
\definecolor{color2}{HTML}{816288}
\definecolor{color3}{HTML}{C6926C}
\definecolor{color4}{HTML}{D5BEA9}
\definecolor{color5}{HTML}{DBCAD3}


%
\usepackage{tikz}
%
\usetikzlibrary{mindmap}
%
\usetikzlibrary{shadows}

\usepackage[hidelinks, pdfencoding = auto]{hyperref}

\usetikzlibrary{backgrounds}


% Information boxes
\newcommand*{\info}[4][16.3]{%
  \node [ annotation, #3, scale=0.65, text width = #1em,
          inner sep = 2mm ] at (#2) {%
  \list{$\bullet$}{\topsep=0pt\itemsep=0pt\parsep=0pt
    \parskip=0pt\labelwidth=8pt\leftmargin=8pt
    \itemindent=0pt\labelsep=2pt}%
    #4
  \endlist
  };
}

\usepackage{wallpaper}

\begin{document}

\ThisCenterWallPaper{}{A}

\centering
\begin{tikzpicture}

	\begin{scope}
[
mindmap,
every node/.style =
{
	concept,
	execute at begin node = \hskip 0pt
},
root concept/.append style =
{
	concept color = color0,
	fill = color5!25,
	line width = 0.5em,
	text width = 16em,
	text = color0,
	font = \huge\scshape
}
]
\node [root concept] (Un_modele_de_cout_pour_le_NoSQL)
at (0em, 0em) {\href{run:PPT1.pdf}{Un modèle de coût pour le NoSQL}};
\end{scope}

	\begin{scope}
		[
			mindmap,
			text = color0,
			grow cyclic,
			every node/.style =
				{
					concept,
					concept color = color1!50,
					text = color0,
					execute at begin node = \hskip 0pt
				},
			root concept/.append style =
				{
					concept color = color1,
					fill = color1!25,
					line width = 0.5em,
					text width = 12em,
					text = color0,
					font = \LARGE\scshape
				},
			level 1/.append style =
				{
					concept color = color1,
					font = \Large\bfseries,
					text width = 10em,
					inner sep = 0pt
				},
			level 2/.append style =
				{
					concept color = color1!75,
					font = \large\bfseries,
					text width = 8em
				}
		]

		\node [root concept] (Evaluation_de_la_Base_de_Donnees_NoSQL)
		at (-25em, 25em)  {\href{run:PPT2.pdf}{Évaluation de la Base de Données NoSQL}}
		child [grow = 90, level distance = 20em, concept color = color1!75]
			{
				node [concept color = color1!75] (Evaluation_des_Attributs_de_Qualite) {Évaluation des Attributs de Qualité}
			}
		child [grow = 180, level distance = 20em, concept color = color1!75]
			{
				node [concept color = color1!75] (Apercu_de_la_Performance) {Aperçu de la Performance}
			}
		;
	\end{scope}

	\begin{scope}
		[
			mindmap,
			text = color0,
			grow cyclic,
			every node/.style =
				{
					concept,
					concept color = color2!50,
					text = color0,
					execute at begin node = \hskip 0pt
				},
			root concept/.append style =
				{
					concept color = color2,
					fill = color2!25,
					line width = 0.5em,
					text width = 12em,
					text = color0,
					font = \LARGE\scshape
				},
			level 1/.append style =
				{
					concept color = color2,
					font = \Large\bfseries,
					text width = 10em,
					inner sep = 0pt
				},
			level 2/.append style =
				{
					concept color = color2!75,
					font = \large\bfseries,
					text width = 8em
				}
		]
		\node [root concept] (Transformation_&_Mirgation)
		at (-25em, -25em)  {\href{run:PPT3.pdf}{Transformation \& Mirgation}}
		child [grow = 0, level distance = 20em, concept color = color2!75]
			{
				node [concept color = color2!75] (Model_Driven_Architecture_MDA) {Model Driven Architecture (MDA)}
				%child [grow = 0, level distance = 15em]
				%{
				%node[concept] (Niveau_Metamodele) {Niveau Métamodèle}										
				%}
				child [grow = 40, level distance = 15em, concept color = color2!50]
					{
						node[concept color = color2!50] (UML_a_NoSQL) {UML à NoSQL}
					}
				child [grow = 0, level distance = 15em, concept color = color2!50]
					{
						node[concept color = color2!50] (UML_a_HBase) {UML à HBase}
					}
				child [grow = -40, level distance = 15em, concept color = color2!50]
					{
						node[concept color = color2!50] (UML_a_GraphDB) {UML à GraphDB}
					}
			}
		child [grow = 90, level distance = 20em, concept color = color2!75]
			{
				node [concept color = color2!75] (Mappage_de_Schema) {Mappage de Schéma}
				child [grow = 90, level distance = 15em, concept color = color2!50]
					{
						node[concept color = color2!50] (SQL_a_NoSQL) {SQL à NoSQL}
					}
				child [grow = 50, level distance = 15em, concept color = color2!50]
					{
						node[concept color = color2!50] (SQLCMS_a_NoSQL) {SQL(CMS) à NoSQL}
					}
				child [grow = 130, level distance = 15em, concept color = color2!50]
					{
						node[concept color = color2!50] (SQL_a_HBase) {SQL à \\HBase}
					}
				child [grow = -30, level distance = 15em, concept color = color2!50]
					{
						node[concept color = color2!50] (SQL_a_MongDB) {SQL à MongDB}
					}
			}
		child [grow = 135, level distance = 20em, concept color = color2!75]
			{
				node [concept color = color2!75] (Cadre_de_Migration) {Cadre de Migration}
				child [grow = 135, level distance = 15em, concept color = color2!50]
					{
						node[concept color = color2!50] (MySQL_a_MongoDB) {MySQL à \\MongoDB}
					}
			}
		child [grow = -45, level distance = 20em, concept color = color2!75]
			{
				node [concept color = color2!75] (Plates-formes_Middleware) {Plates-formes Middleware}
				child [grow = -45, level distance = 15em, concept color = color2!50]
					{
						node[concept color = color2!50] (Cassandra_&_MongoDB) {Cassandra \& MongoDB}
					}
			}
		;
	\end{scope}

	% Applied area: theoretical physics and its subfields

	% Applied area: biology and its subfields



	\begin{scope}
		[
			mindmap,
			text = color0,
			grow cyclic,
			every node/.style =
				{
					concept,
					concept color = color3!50,
					text = color0,
					execute at begin node = \hskip 0pt
				},
			root concept/.append style =
				{
					concept color = color3,
					fill = color3!25,
					line width = 0.5em,
					text width = 12em,
					text = color0,
					font = \LARGE\scshape
				},
			level 1/.append style =
				{
					concept color = color3,
					font = \Large\bfseries,
					text width = 10em,
					inner sep = 0pt
				},
			level 2/.append style =
				{
					concept color = color3!75,
					font = \large\bfseries,
					text width = 8em
				}
		]

		\node [root concept] (Modele_de_Cout)
		at (25em, 25em) {Modèle de Coût}
		child [grow = -90, level distance = 20em, concept color = color3!75]
			{
				node [concept color = color3!75] (Description_Langage) {Description Langage}
				child [grow = -90, level distance = 15em, concept color = color3!50]
					{
						node[concept] (CostDL) {CostDL}
					}
			}
		child [grow = 135, level distance = 20em, concept color = color3!75]
			{
				node [concept color = color3!75] (Modele_de_Cout_generique) {Modèle de Coût générique}
				child [grow = 135, level distance = 15em, concept color = color3!50]
					{
						node[concept] (Systemes_de_Memoire_hierarchiques) {Systèmes de Mémoire hiérarchiques}
					}
				child [grow = 175, level distance = 15em, concept color = color3!50]
					{
						node[concept] (Diversite_et_Interaction) {Diversité et Interaction}
					}
				child [grow = 95, level distance = 15em, concept color = color3!50]
					{
						node[concept] (Environnement_d_Entreposage) {Environnement d'Entreposage}
					}
			}
		child [grow = -45, level distance = 20em, concept color = color3!75]
			{
				node [concept color = color3!75] (Construction_d_un_magasin_de_donnees) {Construction d'un magasin de données}
				child [grow = -45, level distance = 15em, concept color = color3!50]
					{
						node[concept] (Magasins_de_Donnees_distribues_et_repliques) {Magasins de Données distribués et répliqués}
					}
			}
		child [grow = 0, level distance = 20em, concept color = color3!75]
			{
				node [concept color = color3!75] (Apprentissage_automatique) {Apprentissage automatique}
				child [grow = 0, level distance = 15em, concept color = color3!50]
					{
						node[concept] (Temps_et_Ressources_consommes) {Temps et Ressources consommés}
					}
			}
		child [grow = 90, level distance = 20em, concept color = color3!75]
			{
				node [concept color = color3!75] (Complexite_structurelle) {Complexité structurelle}
				child [grow = 90, level distance = 15em, concept color = color3!50]
					{
						node[concept] (JSON_&_MongoDB) {JSON \& MongoDB}
					}
			}
		child [grow = 45, level distance = 20em, concept color = color3!75]
			{
				node [concept color = color3!75] (Expressivite_et_Complexite_des_Requetes) {Expressivité et Complexité des Requêtes}
				child [grow = 45, level distance = 15em, concept color = color3!50]
					{
						node[concept] (MongoDB) {MongoDB}
					}
			}
		;
	\end{scope}




	\begin{scope}
		[
			mindmap,
			text = color0,
			grow cyclic,
			every node/.style =
				{
					concept,
					concept color = color4!50,
					text = color0,
					execute at begin node = \hskip 0pt
				},
			root concept/.append style =
				{
					concept color = color4,
					fill = color4!25,
					line width = 0.5em,
					text width = 12em,
					text = color0,
					font = \LARGE\scshape
				},
			level 1/.append style =
				{
					concept color = color4,
					font = \Large\bfseries,
					text width = 10em,
					inner sep = 0pt
				},
			level 2/.append style =
				{
					concept color = color4!75,
					font = \large\bfseries,
					text width = 8em
				}
		]

		\node [root concept] (Gestion_des_Donnees)
		at (25em, -25em) {Gestion des Données}
		child [grow = -45, level distance = 20em, concept color = color4!75]
			{
				node [concept color = color4!75] (Systeme_de_Gestion_des_Metadonnees) {Système de Gestion des Métadonnées}
				child [grow = -45, level distance = 15em, concept color = color4!50]
					{
						node[concept color = color4!50] (Gestion_de_Multiples) {Gestion de Multiples}
					}
			}
		child [grow = 0, level distance = 20em, concept color = color4!75]
			{
				node [concept color = color4!75] (Traitement_parallele_des_Donnees_JSON) {Traitement parallèle des Données JSON}
				child [grow = 0, level distance = 15em, concept color = color4!50]
					{
						node[concept color = color4!50] (Regles_de_Reecriture) {Règles de Réécriture}
					}
			}
		child [grow = -90, level distance = 20em, concept color = color4!75]
			{
				node [concept color = color4!75] (Requetes_independantes_du_Schema) {Requêtes indépendantes du Schéma}
				child [grow = -90, level distance = 15em, concept color = color4!50]
					{
						node[concept color = color4!50] (Documents_multi-structures) {Documents multi-structurés}
					}
			}
		;
	\end{scope}




\end{tikzpicture}

\end{document}
