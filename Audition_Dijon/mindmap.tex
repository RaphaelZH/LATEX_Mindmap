% Use the standalone as the document class with an option which allows specifying the border.
\documentclass[border = 10pt]{standalone}

% Allow to entering comfortably English.
\usepackage[T1]{fontenc}
\usepackage[utf8]{inputenc}
\usepackage[french]{babel}

% 
\usepackage{xcolor}
\definecolor{color0}{HTML}{372639}
\definecolor{color1}{HTML}{C2A4C0}
\definecolor{color2}{HTML}{816288}
\definecolor{color3}{HTML}{C6926C}
\definecolor{color4}{HTML}{D5BEA9}
\definecolor{color5}{HTML}{DBCAD3}


%
\usepackage{tikz}
%
\usetikzlibrary{mindmap}
%
\usetikzlibrary{shadows}

\usepackage[hidelinks, pdfencoding = auto]{hyperref}

\usetikzlibrary{backgrounds}

\usepackage{wallpaper}

\begin{document}
	
%\ThisCenterWallPaper{}{./figures/Social_Network_Analysis_Visualization_transparent.png}

\centering

\begin{tikzpicture}

\begin{scope}
	[
		mindmap,
		every node/.style =
		{
			concept,
			execute at begin node = \hskip 0pt
		},
		root concept/.append style =
		{
			concept color = color0,
			rectangle,
			rounded corners = 2em,
			fill = color5!10,
			line width = 0.5em,
			text width = 40em,
			text = color0,
			font = \huge\scshape,
			inner sep = 2em
		}
	]
	\node [root concept] (Nom de projet)
	at (0em, 0em) {{SYSTÈMES INTELLIGENTS \\POUR LA TRANSMISSION DES \\HUMANITÉS NUMÉRIQUES ET \\POUR LA RECHERCHE EN SANTÉ}};
\end{scope}

\begin{scope}
	[
		mindmap,
		text = color0,
		grow cyclic,
		every node/.style =
			{
				concept,
				concept color = color1!50,
				text = color0,
				execute at begin node = \hskip 0pt
			},
		root concept/.append style =
			{
				concept color = color1,
				draw = color1!75,
				fill = color1!10,
				line width = 1em,
				text width = 20em,
				text = color0,
				font = \LARGE\scshape,
				inner sep = 0pt
			},
		level 1/.append style =
			{
				concept color = color1,
				draw = color1!50,
				fill = color1!5,
				line width = 0.5em,
				text width = 18em,
				text = color0,
				font = \Large\bfseries,
				inner sep = 0pt
			},
		level 2/.append style =
			{
				concept color = color1,
				draw = color1!50,
				fill = color1!5,
				line width = 0.5em,
				text width = 16em,
				text = color0,
				font = \large\bfseries,
				inner sep = 0pt
			}
	]

	\node [root concept, text width = 20em] (UBFC)
		at (-30em, 30em)
			{\includegraphics[width = 0.95\textwidth]{./logos/logo UBFC.png}}
		child [grow = 180, level distance = 30em, concept color = color1!50]
			{
				node [level 1, text width = 18em] (UFC) {\includegraphics[width = 0.9\textwidth]{./logos/logo UFC.png}}
			}
		child [grow = 45, level distance = 30em, concept color = color1!50]
			{
				node [level 1, text width = 18em, inner sep = -0.5em] (I-SITE BFC) {\includegraphics[width = 0.65\textwidth]{./logos/logo I-SITE BFC.png}}
			}
		child [grow = 90, level distance = 30em, concept color = color1!50]
			{
				node [level 1, text width = 18em, inner sep = -1em] (LLC) {\LARGE{Pôle Lettres, \\Langues et \\Communication} \\\vspace{1em} \Huge{- LLC}}
				child [grow = 45, level distance = 30em, concept color = color1!50]
				{
					node [level 2, text width = 16em] (LECLA) {\includegraphics[width = 0.95\textwidth]{./logos/logo LECLA.png}}
				}
				child [grow = 90, level distance = 30em, concept color = color1!50]
				{
					node [level 2, text width = 16em, inner sep = -1em] (MSH) {\includegraphics[width = 0.85\textwidth]{./logos/logo MSH.png}}
				}
				child [grow = 135, level distance = 30em, concept color = color1!50]
				{
					node [level 2, text width = 16em, inner sep = -0.75em] (ELLIADD) {\includegraphics[width = 0.8\textwidth]{./logos/logo ELLIADD.png}}
					child [grow = 90, level distance = 30em, concept color = color1!50]
					{
						node [level 2, text width = 16em, inner sep = -1.5em] (HULIN) {\includegraphics[width = \textwidth]{./profile/HULIN.png}}
					}
				}
			}
		;
	\begin{pgfonlayer}{background}
		\draw [densely dashed] 
			(Nom de projet)
				edge [color = color1!50, line width = 1em] 
					node[
						solid,
						annotation,
						right, 
						concept color = color1,
						draw = color1!50,
						fill = color1!5,
						line width = 0.25em,
						text width = 20em,
						text = color0,
						font = \LARGE\bfseries,
						inner sep = 1em
					] {Ce projet s'inscrit naturellement dans l'axe 1 de l'I-SITE consacré aux systèmes intelligents, ainsi que dans l'axe 3 sur la santé grâce à l'approche comparative des usages.}
				(I-SITE BFC)
				edge [color = color1!50, line width = 1em] 
						node[
						solid,
						annotation,
						right, 
						concept color = color1,
						draw = color1!50,
						fill = color1!5,
						line width = 0.25em,
						text width = 20em,
						text = color0,
						font = \LARGE\bfseries,
						inner sep = 1em
					] {Ce projet s'inscrit dans le paradigme de la transmission des pôles thématiques de la fédération des MSH BFC.}
				(MSH)
				edge [color = color1!50, line width = 1em] 
					node[
						solid,
						annotation,
						right, 
						concept color = color1,
						draw = color1!50,
						fill = color1!5,
						line width = 0.25em,
						text width = 20em,
						text = color0,
						font = \LARGE\bfseries,
						inner sep = 1em
					] {Ce projet rattaché au laboratoire ELLIADD, 7 chercheurs ELLIADD travaillent déjà au projet HUMANE.}
				(ELLIADD)
				edge [color = color1!50, line width = 1em] 
					node[
						solid,
						annotation,
						right, 
						concept color = color1,
						draw = color1!50,
						fill = color1!5,
						line width = 0.25em,
						text width = 20em,
						text = color0,
						font = \LARGE\bfseries,
						inner sep = 1em
					] {Ce projet concerne tous les pôles de UBFC : la valorisation des corpus dans le champ de l’enseignement des HN mobilisera les pôles AL, DTMS et LLC ; l’expérience en HN et en web sémantique mobilise le pôle CCM et sur son axe transversal SEISM portant sur la recherche en éducation ; le design de la plateforme et son évaluation, le pôle ERCOS.}
				(UBFC)
				edge [color = color1!50, line width = 1em] 
					node[
						solid,
						annotation,
						right, 
						concept color = color1,
						draw = color1!50,
						fill = color1!5,
						line width = 0.25em,
						text width = 20em,
						text = color0,
						font = \LARGE\bfseries,
						inner sep = 1em
					] {Cette offre de thèse proposée par le laboratoire ELLIADD et l’école doctorale LECLA.}
				(LECLA)
		;
	\end{pgfonlayer}
\end{scope}

\begin{scope}
	[
		mindmap,
		text = color0,
		grow cyclic,
		every node/.style =
		{
			concept,
			concept color = color2!50,
			text = color0,
			execute at begin node = \hskip 0pt
		},
		root concept/.append style =
		{
			concept color = color2,
			draw = color2!75,
			fill = color2!10,
			line width = 1em,
			text width = 20em,
			text = color0,
			font = \LARGE\scshape,
			inner sep = 0pt
		},
		level 1/.append style =
		{
			concept color = color2,
			draw = color2!50,
			fill = color2!5,
			line width = 0.5em,
			text width = 18em,
			text = color0,
			font = \Large\bfseries,
			inner sep = 0pt
		},
		level 2/.append style =
		{
			concept color = color2,
			draw = color2!50,
			fill = color2!5,
			line width = 0.5em,
			text width = 16em,
			text = color0,
			font = \large\bfseries,
			inner sep = 0pt
		}
	]

	\node [root concept, text width = 20em, inner sep = -0.5em] (HUMANE)
		at (30em, -30em)
			{\Huge{Projet \\HUMANE} \\\vspace{0.5em}\Large{\og Humanités Numériques \\pour l'Éducation \fg}}
		child [grow = 180, level distance = 30em, concept color = color2!50]
			{
				node [level 1, text width = 18em, inner sep = -0.5em] (MENJS) {\includegraphics[width = 0.8\textwidth]{./logos/logo MENJS.png}}
			}
		child [grow = 45, level distance = 30em, concept color = color2!50]
			{
				node [level 1, text width = 18em] (GIS 2IF) {\includegraphics[width = 0.9\textwidth]{./logos/logo GIS 2IF.png}}
			}
		child [grow = 0, level distance = 30em, concept color = color2!50]
			{
				node [level 1, text width = 18em] (canope) {\includegraphics[width = 0.9\textwidth]{./logos/logo canope.png}}
			}
		child [grow = -45, level distance = 30em, concept color = color2!50]
			{
				node [level 1, text width = 18em, inner sep = -2em] (humanistica) {\includegraphics[width = 0.9\textwidth]{./logos/logo humanistica.png}}
			}
		;
	\begin{pgfonlayer}{background}
	\draw [densely dashed] 
	(Nom de projet)
	edge [color = color2!50, line width = 1em] 
	node[
	solid,
	annotation,
	right, 
	concept color = color2,
	draw = color2!50,
	fill = color2!5,
	line width = 0.25em,
	text width = 20em,
	text = color0,
	font = \LARGE\bfseries,
	inner sep = 1em
	] {Ce projet s'inscrit naturellement dans l'axe 1 de l'I-SITE consacré aux systèmes intelligents, ainsi que dans l'axe 3 sur la santé grâce à l'approche comparative des usages.}
	(I-SITE BFC)
	edge [color = color2!50, line width = 1em] 
	node[
	solid,
	annotation,
	right, 
	concept color = color2,
	draw = color2!50,
	fill = color2!5,
	line width = 0.25em,
	text width = 20em,
	text = color0,
	font = \LARGE\bfseries,
	inner sep = 1em
	] {Ce projet s'inscrit dans le paradigme de la transmission des pôles thématiques de la fédération des MSH BFC.}
	(MSH)
	edge [color = color2!50, line width = 1em] 
	node[
	solid,
	annotation,
	right, 
	concept color = color2,
	draw = color2!50,
	fill = color2!5,
	line width = 0.25em,
	text width = 20em,
	text = color0,
	font = \LARGE\bfseries,
	inner sep = 1em
	] {Ce projet rattaché au laboratoire ELLIADD, 7 chercheurs ELLIADD travaillent déjà au projet HUMANE.}
	(ELLIADD)
	edge [color = color2!50, line width = 1em] 
	node[
	solid,
	annotation,
	right, 
	concept color = color2,
	draw = color2!50,
	fill = color2!5,
	line width = 0.25em,
	text width = 20em,
	text = color0,
	font = \LARGE\bfseries,
	inner sep = 1em
	] {Ce projet concerne tous les pôles de UBFC : la valorisation des corpus dans le champ de l’enseignement des HN mobilisera les pôles AL, DTMS et LLC ; l’expérience en HN et en web sémantique mobilise le pôle CCM et sur son axe transversal SEISM portant sur la recherche en éducation ; le design de la plateforme et son évaluation, le pôle ERCOS.}
	(UBFC)
	edge [color = color2!50, line width = 1em] 
	node[
	solid,
	annotation,
	right, 
	concept color = color2,
	draw = color2!50,
	fill = color2!5,
	line width = 0.25em,
	text width = 20em,
	text = color0,
	font = \LARGE\bfseries,
	inner sep = 1em
	] {Cette offre de thèse proposée par le laboratoire ELLIADD et l’école doctorale LECLA.}
	(LECLA)
	;
	\end{pgfonlayer}
\end{scope}


\begin{scope}
	[
		mindmap,
		text = color0,
		grow cyclic,
		every node/.style =
		{
			concept,
			concept color = color3!50,
			text = color0,
			execute at begin node = \hskip 0pt
		},
		root concept/.append style =
		{
			concept color = color3,
			draw = color3!75,
			fill = color3!10,
			line width = 1em,
			text width = 20em,
			text = color0,
			font = \LARGE\scshape,
			inner sep = 0pt
		},
		level 1/.append style =
		{
			concept color = color3,
			draw = color3!50,
			fill = color3!5,
			line width = 0.5em,
			text width = 18em,
			text = color0,
			font = \Large\bfseries,
			inner sep = 0pt
		},
		level 2/.append style =
		{
			concept color = color3,
			draw = color3!50,
			fill = color3!5,
			line width = 0.5em,
			text width = 16em,
			text = color0,
			font = \large\bfseries,
			inner sep = 0pt
		}
	]
	
	\node [root concept, text width = 20em] (Co-tutelle Internationale)
		at (30em, 30em)
			{\Huge{Co-tutelle} \\\huge{Internationale}}
		child [grow = 0, level distance = 30em, concept color = color3!50]
			{
				node [level 1, text width = 18em] (Laval University) {\includegraphics[width = 0.95\textwidth]{./logos/logo Laval University.png}}
				child [grow = 0, level distance = 30em, concept color = color3!50]
				{
					node [level 2, text width = 18em, inner sep = 0.25em] (LAVAL FSE) {\includegraphics[width = 0.95\textwidth]{./logos/loge LAVAL FSE.png}}
				}
			}
		child [grow = 45, level distance = 30em, concept color = color3!50]
			{
				node [level 1, text width = 18em, inner sep = -2em] (RPI) {\includegraphics[width = 0.95\textwidth]{./logos/logo RPI.png}}
				child [grow = 0, level distance = 30em, concept color = color3!50]
				{
					node [level 2, text width = 18em, fill = color3!75, inner sep = 0.25em] (RPI IDEA) {\includegraphics[width = 0.95\textwidth]{./logos/logo RPI IDEA.png}}
				}
				child [grow = 45, level distance = 30em, concept color = color3!50]
				{
					node [level 2, text width = 16em, inner sep = -1.5em] (McGuinness) {\includegraphics[width = \textwidth]{./profile/McGuinness.png}}
				}
				child [grow = 90, level distance = 30em, concept color = color3!50]
				{
					node [level 2, text width = 16em, inner sep = -1.5em] (McCusker) {\includegraphics[width = \textwidth]{./profile/McCusker.png}}
				}
			}
		;%Co-directrice Deborah Louise McGuinness 
		%
	\begin{pgfonlayer}{background}
	\draw [densely dashed] 
	(Nom de projet)
	edge [color = color3!50, line width = 1em] 
	node[
	solid,
	annotation,
	right, 
	concept color = color3,
	draw = color3!50,
	fill = color3!5,
	line width = 0.25em,
	text width = 20em,
	text = color0,
	font = \LARGE\bfseries,
	inner sep = 1em
	] {Ce projet s'inscrit naturellement dans l'axe 1 de l'I-SITE consacré aux systèmes intelligents, ainsi que dans l'axe 3 sur la santé grâce à l'approche comparative des usages.}
	(I-SITE BFC)
	edge [color = color3!50, line width = 1em] 
	node[
	solid,
	annotation,
	right, 
	concept color = color3,
	draw = color3!50,
	fill = color3!5,
	line width = 0.25em,
	text width = 20em,
	text = color0,
	font = \LARGE\bfseries,
	inner sep = 1em
	] {Ce projet s'inscrit dans le paradigme de la transmission des pôles thématiques de la fédération des MSH BFC.}
	(MSH)
	;
	\end{pgfonlayer}
\end{scope}

\begin{scope}
[
mindmap,
text = color0,
grow cyclic,
every node/.style =
{
	concept,
	concept color = color4!50,
	text = color0,
	execute at begin node = \hskip 0pt
},
root concept/.append style =
{
	concept color = color4,
	draw = color4!75,
	fill = color4!10,
	line width = 1em,
	text width = 20em,
	text = color0,
	font = \LARGE\scshape,
	inner sep = 0pt
},
level 1/.append style =
{
	concept color = color4,
	draw = color4!50,
	fill = color4!5,
	line width = 0.5em,
	text width = 18em,
	text = color0,
	font = \Large\bfseries,
	inner sep = 0pt
},
level 2/.append style =
{
	concept color = color4,
	draw = color4!50,
	fill = color4!5,
	line width = 0.5em,
	text width = 16em,
	text = color0,
	font = \large\bfseries,
	inner sep = 0pt
}
]

	\node [level 1, text width = 20em] (1)
		at (-40em, -20em)
			{\Huge{Étape 1 \\état de l’art \\1/09/21 \\- \\20/12/21}}
		child [grow = 0, level distance = 30em, concept color = color4!50]
			{
				\node [level 1, text width = 20em] (2) {\Huge{Étape 1 \\état de l’art \\1/09/21 \\- \\20/12/21}}
				child [grow = 0, level distance = 30em, concept color = color4!50]
				{
					node [level 1, text width = 18em, inner sep = -0.5em] (I-SITE BFC) {\includegraphics[width = 0.65\textwidth]{./logos/logo I-SITE BFC.png}}
					child [grow = 0, level distance = 30em, concept color = color4!50]
					{
						node [level 1, text width = 18em, inner sep = -0.5em] (I-SITE BFC) {\includegraphics[width = 0.65\textwidth]{./logos/logo I-SITE BFC.png}}
						child [grow = 0, level distance = 30em, concept color = color4!50]
						{
							node [level 1, text width = 18em, inner sep = -0.5em] (I-SITE BFC) {\includegraphics[width = 0.65\textwidth]{./logos/logo I-SITE BFC.png}}
						}
					}
				}
			}
		;
\end{scope}

\end{tikzpicture}

\end{document}
